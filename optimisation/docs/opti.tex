\documentclass[10pt,a4paper]{report}
%\usepackage[latin1]{inputenc}
\usepackage[utf8]{inputenc}
\usepackage{amsmath}
\usepackage{amsfonts}
\usepackage{amssymb}
\usepackage{ragged2e}
\usepackage{graphicx}
\usepackage{fixltx2e}
\usepackage{multicol}
\usepackage{tabularx}
\usepackage{tikz}
\usepackage{hyperref}
\hypersetup{
    colorlinks=true,
    linkcolor=blue,
    filecolor=magenta,      
    urlcolor=blue,
    }
\usepackage{tabularx}
\usepackage{mathtools}
\usepackage{tikz}
\usetikzlibrary{arrows,shapes,automata,petri,positioning,calc}
\usepackage{hyperref}
\usepackage{tikz}
\usetikzlibrary{matrix,calc}
\usepackage[margin=0.5in]{geometry}
% ---- power functions -----% 
\newcommand{\myvec}[1]{\ensuremath{\begin{pmatrix}#1\end{pmatrix}}}
\let\vec\mathbf

\providecommand{\norm}[1]{\left\lVert#1\right\rVert}
\providecommand{\abs}[1]{\left\vert#1\right\vert}
\let\vec\mathbf

\newcommand{\mydet}[1]{\ensuremath{\begin{vmatrix}#1\end{vmatrix}}}
\providecommand{\brak}[1]{\ensuremath{\left(#1\right)}}
\providecommand{\lbrak}[1]{\ensuremath{\left(#1\right.}}
\providecommand{\rbrak}[1]{\ensuremath{\left.#1\right)}}
\providecommand{\sbrak}[1]{\ensuremath{{}\left[#1\right]}}
%-------end power functions----%
\newenvironment{Figure}
  {\par\medskip\noindent\minipage{\linewidth}}
  {\endminipage\par\medskip}
\begin{document}
\begin{figure*}[!tbp]
  \centering
  \begin{minipage}[b]{0.4\textwidth}
%   \includegraphics[scale=0.05]{logo.jpg} 
  \end{minipage}
  \hfill
  \vspace{5mm}\begin{minipage}[b]{0.4\textwidth}
%\raggedleft \includegraphics[scale=0.6]{nrc.jpeg} 
  \end{minipage}\vspace{0.2cm}
\end{figure*}
\raggedright \textbf{Name}:\hspace{1mm} T. Sai Raghavendra\hspace{3cm} \Large \textbf{Optimization}\hspace{2.5cm} % 
\normalsize \textbf{Roll No.} :\hspace{1mm} FWC22087\vspace{1cm}
\begin{multicols}{2}
\textbf{Problem Statement:}\vspace{2mm}
\justify Let f(x) = $sin^3 x+\lambda sin^2 x, \frac{-\pi}{2} < x < \frac{\pi}{2} $. Find the intervals in which $\lambda$ should lie in order that f(x) has exactly one minimum and exactly one maximum.
\\
\\
\vspace{4mm}
\textbf{Solution:}
\\
Given function is ,
	\begin{align}
	\label{eq:one}
	f(x)= sin^3 x+\lambda sin^2 x
	\end{align}
	\textbf{Theoritical proof:}
\vspace{4mm}
\\
Let y=f(x)=$sin^3 x+\lambda sin^2 x, \frac{-\pi}{2} < x < \frac{\pi}{2} $\\ \\
Let $sinx=t$
    \begin{equation}
	 \frac{dy}{dt}=3t^2+2t\lambda=t(3t+2\lambda)
	 \end{equation}
	 for exactly one minima and exactly one maxima  $\frac{dy}{dx} $ must have two distinct roots $\in$ (-1,1) \\ \\
	 \hspace*{2.2cm}$t=0$ and $t=\frac{-2\lambda}{3}\in (-1,1)$\\
	 \begin{align}
	 -1<\frac{-2\lambda}{3}<1\\
	 \frac{-3}{2}<\lambda<\frac{3}{2}\\
	 \lambda\in (\frac{-3}{2},\frac{3}{2})
	 \end{align}
	\textbf{Objective function:}
	\begin{align}
	\min_xf(x)= sin^3 x+\lambda sin^2 x, \frac{-\pi}{2} < x < \frac{\pi}{2} \\
	\max_xf(x)= sin^3 x+\lambda sin^2 x, \frac{-\pi}{2} < x < \frac{\pi}{2} 
        \end{align}
	\textbf{constraints:}\\
	\begin{align}
		x \in \{ \frac{-\pi}{2}, \frac{\pi}{2}\} 
	\end{align}
\textbf{Calculation of Minima using gradient descent algorithm:}
\justify{Minima of the above \eqref{eq:one}, can be calculated from the following expression,}
Differentiating \eqref{eq:ten} yields,
\begin{align}
       \boxed{x_{n+1} = x_n - \alpha \nabla h(x_n)} 
\end{align}
\begin{align}
\label{eq:ten}
f(x)= sin^3 x+\lambda sin^2 x \\ 
\nabla f(x) = sinx cosx (3sinx+2\lambda)
\end{align}
\vspace{1mm}
Taking $x_0=\frac{-\pi}{2},\alpha=0.0001$ and precision = 0.000000001, values obtained using python are:
    \begin{align}
        \boxed{\text{Minima} =-2.5}\\     
        \boxed{\text{Minima Point} = -1.5708}
    \end{align}
   
\textbf{Calculation of Maxima using gradient ascent algorithm:}
\justify{Maxima of the above \eqref{eq:one}, can be calculated from the following expression,}
Differentiating \eqref{eq:fifteen} yields,
\begin{align}
       \boxed{x_{n+1} = x_n - \alpha \nabla h(x_n)} 
\end{align}
\begin{align}
\label{eq:fifteen}
f(x)= sin^3 x+\lambda sin^2 x \\ 
\nabla f(x) = sinx cosx (3sinx+2\lambda)
\end{align}
\vspace{1mm}
Taking $x_0=\frac{\pi}{2},\alpha=0.0001$ and precision = 0.000000001, values obtained using python are:
    \begin{align}
        \boxed{\text{Maxima} =2.5}\\     
        \boxed{\text{Maxima Point} = 1.5707}
    \end{align}
\vspace{2mm} 
\\
\raggedright $\vec{Plots:}$ \\
\includegraphics[width=\columnwidth]{fig.jpg}
\label{fig:Figure}
\includegraphics[width=\columnwidth]{fig1.jpg}
\label{fig:Figure}
\textbf{Conclusion:}\\
1. At first, the given function has been differentiated to find h'(x).
\\
2. Later, the given function h(x) is solved by gradient   \\descent algorithm to find minima and the point at which h(x) is minimum.
Download the code to execute the above problem statement.
\vspace{4mm}
\\
\boxed{\href{https://github.com/Sairaghavendra36/Fwc-2022}{https://github.com/Sairaghavendra36/Fwc-2022}.}
\end{multicols}
\end{document}

