\documentclass[journal,12pt,twocolumn]{IEEEtran}
\usepackage{graphicx}
\graphicspath{{./figs/}}{}
\usepackage{amsmath,amssymb,amsfonts,amsthm}
\newcommand{\myvec}[1]{\ensuremath{\begin{pmatrix}#1\end{pmatrix}}}
\usepackage{listings}
\usepackage{tabularx}
\usepackage{watermark}
\usepackage{titlesec}
\let\vec\mathbf
\lstset{
frame=single, 
breaklines=true,
columns=fullflexible
}
\thiswatermark{\centering \put(0,-105.0){\includegraphics[scale=0.07]{logo.jpg}} }

\title{\mytitle}
\title{
Matrix - Line Assignment
}
\author{T.Sai Raghavendra}
\begin{document}
\maketitle
\tableofcontents
\bigskip


\section{\textbf{Problem}}
The vertices of a triangle are $ [at_1t_2,a(t_1+t_2)],[at_2t_3,a(t_2+t_3)],[at_3t_1,a(t_3+t_1)] $. Find the orthocentre of the triangle.\\


\section{\textbf{Solution}}
Orthocenter of a triangle is the point where perpendiculars drawn to the opposite side from each vertex of the triangle intersect.   \\

To find the orthocenter first we find the equation of line AP which is given by\\

\begin{equation}
 \vec{m_1^{\top}}(\vec{{x}-{A}}) = 0  
\end{equation}
where $\vec{m_1} = \vec{(B-C)}$ \\
Similarly the equation of line AP is given by\\
\begin{equation}
 \vec{m_2^{\top}}(\vec{{x}-B}) = 0   \label{eq-3}
\end{equation}
where $\vec{m_2} = \vec{(B-C)}$ \\
By Solving eq1 and eq2 we get two line equations are represented in matrix form \\
$\myvec{\vec{at_3(t_2-t_1)} & \vec{a(t_2-t_1)} & \vec{-a^2(t_2-t_1)[t_1t_2t_3 + (t_1+t_2)]} \\ \vec{at_1(t_2-t_3)} & \vec{a(t_2-t_3)} & \vec{-a^2(t_2-t_3)[t_1t_2t_3 + (t_2+t_3)]}}$\\
$R_1 \rightarrow \frac{1}{a(t_2-t_1)}$ 
$R_2 \rightarrow \frac{1}{a(t_2-t_3)}$\\
$\myvec{\vec{t_3} & 1 & \vec{-a[t_1t_2t_3 + (t_1+t_2)} \\ \vec{t_1}& 1 &\vec{-a[t_1t_2t_3 + (t_2+t_3)}}$\\
 $R_1 \rightarrow {R_1-R_2}$\\
$\myvec{\vec{(t_3-t_1)} & 0 & \vec{a(t_3-t_1)} \\ \vec{t_1}& 1 & \vec{-a[t_1t_2t_3 + (t_2+t_3)}}$\\
$R_1 \rightarrow \frac{1}{(t_3-t_1)}$ \\
$\myvec{1 & 0 & \vec{a} \\ \vec{t_1} & 1 &\vec{-a[t_1t_2t_3 + (t_2+t_3)}}$\\
$R_2 \rightarrow {R_2-t_1R_1}$\\
$\myvec{1 &0 & \vec{a} \\  0& 1 & \vec{-a[t_1t_2t_3 + (t_1+t_2+t_3)}}$\\
By making X and Y Coordinates of eq1 and eq2 as Identity Matrix there obtained Intersection point i.e.., $\vec{Orthocentre}$\\
Therefore the Orthocentre of triangle is\\
$\vec{X} =\myvec{\vec{a}\\ \vec{-a[t_1t_2t_3 + (t_1+t_2+t_3)}}$

\pagebreak\begin{tabular}{|c |c |c|}
    \hline % <-- Alignments: 1st column left, 2nd middle and 3rd right, with vertical lines in between
      \large\textbf{ Symbol } & \large\textbf{ Co-ordinates } & \large\textbf{Description}\\
      \hline
       \large m1 & $\ \begin{pmatrix} at_3(t_2-t_1)\\ a(t_2-t_1) \end{pmatrix}$ & direction vector of m1 \\
       
       \large m2 & $\ \begin{pmatrix} at_1(t_2-t_3)\\ a(t_2-t_3) \end{pmatrix}$ & direction vector of m2 \\
       
		 \large A & $\ \begin{pmatrix} at_1t_2\\ a(t_1+t_2) \end{pmatrix}$ & direction vector of m1 \\  
		 
		 \large B & $\ \begin{pmatrix} at_2t_3\\ a(t_2+t_3) \end{pmatrix}$ & direction vector of m1 \\  
		 
		 \large C & $\ \begin{pmatrix} at_3t_1\\ a(t_3+t_1) \end{pmatrix}$ & direction vector of m1 \\  
  
      \hline
   \end{tabular}




\section{\textbf{Figure}}
\begin{figure}[h]
    \centering
\includegraphics[width=\columnwidth]{fig.jpg}
    \label{fig:my_label}
\end{figure}


\begin{center}
$\vec{IV. Code Link}$
\end{center}
\begin{lstlisting}
https://github.com/Sairaghavendra36/Fwc-2022/blob/main/Matrix/Line/line.py
\end{lstlisting}
Execute the code by using the command\\
\textbf{python3 line.py}




\end{document}