\documentclass[journal,12pt,twocolumn]{IEEEtran}
\usepackage{graphicx}
\graphicspath{{./figs/}}{}
\usepackage{amsmath,amssymb,amsfonts,amsthm}
\newcommand{\myvec}[1]{\ensuremath{\begin{pmatrix}#1\end{pmatrix}}}
\usepackage{listings}
\usepackage{watermark}
\usepackage{titlesec}
\let\vec\mathbf

\titlespacing{\subsection}{0pt}{\parskip}{-3pt}
\titlespacing{\subsubsection}{0pt}{\parskip}{-\parskip}
\titlespacing{\paragraph}{0pt}{\parskip}{\parskip}
\newcommand{\figuremacro}[5]{
    
}
\lstset{
frame=single, 
breaklines=true,
columns=fullflexible
}
\thiswatermark{\centering \put(0,-105.0){\includegraphics[scale=0.08]{logo.jpg}} }

\sloppy
\title{\mytitle}
\title{
Matrix Assignment - Conic
}
\author{T.Sai Raghavendra(FWC22087)}
\begin{document}
\maketitle
\tableofcontents
\bigskip


\section{\textbf{Problem}}
If the line x-1=0 is the directrix of the parabola to $y^2-kx+8=0$ then find one of the values of k is\\
\hspace*{1.5cm} $a)\frac{1}{8} \hspace*{1cm} b)4\hspace*{1cm} c)\frac{1}{4} \hspace*{1cm}d)8$\\


\section{\textbf{Solution}}

Given line, x-1 = 0 \\
we know that the vector equation of the line is \\
\begin{center}
$\vec{n}^\top x = c$
\end{center}

Compare the given line with the vector equation \\

\begin{center}
$\myvec{1&0} \myvec{x\\y}$ = 1 \\
\end{center}

By comparing we get, \\

\begin{center}
$n = \myvec{1\\0}$ , c = 1 \\
\end{center}

From $y^2-kx+8=0$ \\

We know that the equation of a conic with directrix $\vec{n}^\top x = c$, eccentricity e and focus F is given by \\
\begin{center}
$\vec{x^\top Vx}+2\vec{u^\top}x+f = 0$ \\
\end{center}

Compare the given parabola with equation of conic we get,\\

$\vec{V} = \myvec{0&0\\0&1}$ $\vec{u} = \myvec{\frac{-k}{2} \\ \\ 0}$ \vspace*{0.2cm} f = 8\\

If we can find vector $\vec{'u'}$ just by comparing the $\vec{'u'}$ vector we can obtain the $\vec{k}$ value\\ 

To find $\vec{u}$ we have,

\begin{align}
\vec{u}=ce^2\vec{n}-||\vec{n}||^2\vec{F}
\end{align}
To find Focus $\vec{F}$ in equation(1) we have,
\begin{equation}
f = ||\vec{n}||^2||\vec{F}||^2-c^2e^2
\end{equation}
By substituting the values of f,c,e,$\vec{n}$ we get,
\begin{center}
$||\vec{F}||^2 = 9$
\end{center}

\begin{center}
$||\vec{F}||^2$ = $\vec{F^{\top}F}$
\end{center}

After comparing the conic equation with parabola we know that the y co-ordinate of $\vec{u}$ is zero(0) such that, \\


$||\vec{F}||^2$ can be written in two cases as y co-ordinate is zero..,
\begin{center}
case 1: $\myvec{-3&0}$ $\myvec{-3\\0}$ \\
case 2: $\myvec{3&0}$ $\myvec{3\\0}$ \\
\end{center}

From case 1: The $\vec{F}$ is $\myvec{-3\\0}$\\

By substituing all the values of c,e,$\vec{n}$,$\vec{F}$ in equation(1) we get,\\

\begin{center}
$\vec{u} = \myvec{4 \\ 0}$ \\ 
\vspace{0.2cm} we got, k = -8
\end{center}

From case 2: The $\vec{F}$ is $\myvec{3\\0}$\\

By substituing all the values of c,e,$\vec{n}$,$\vec{F}$ in equation(1) we get,\\

\begin{center}
$\vec{u} = \myvec{-2 \\ 0}$ \\ 
\vspace{0.2cm} we got, k = 4
\end{center}

\section{\textbf{Figure}}
\begin{figure}[h]
    \centering
\includegraphics[width=\columnwidth]{fig.jpg}
    \label{fig:my_label}
\end{figure}


\begin{center}
$\vec{IV.\hspace{.2cm} Code Link}$
\end{center}
\begin{lstlisting}
https://github.com/Sairaghavendra36/Fwc-2022/blob/main/Matrix/Conic/Conic.py
\end{lstlisting}
Execute the code by using the command\\
\textbf{python3 line.py}


\end{document}